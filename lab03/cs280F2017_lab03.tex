\documentclass[11pt]{article}

% NOTE: The "Edit" sections are changed for each assignment

% Edit these commands for each assignment

\newcommand{\assignmentduedate}{October 3}
\newcommand{\assignmentassignedate}{September 26}
\newcommand{\assignmentnumber}{Three}

\newcommand{\labyear}{2017}
\newcommand{\labday}{Tuesday}
\newcommand{\labtime}{2:30 pm}

\newcommand{\assigneddate}{Assigned: \labday, \assignmentassignedate, \labyear{} at \labtime{}}
\newcommand{\duedate}{Due: \labday, \assignmentduedate, \labyear{} at \labtime{}}

% Edit these commands to give the name to the main program

\newcommand{\mainprogram}{\lstinline{generator.py}}
\newcommand{\mainprogramsource}{\lstinline{generator/generator.py}}
\newcommand{\maintestsource}{\lstinline{generator/tests/test_generator.py}}

% Edit this commands to describe key deliverables

\newcommand{\reflection}{\lstinline{README.md}}

% Commands to describe key git tasks

\newcommand{\gitcommitfile}[1]{\command{git commit #1}}
\newcommand{\gitaddfile}[1]{\command{git add #1}}

\newcommand{\gitadd}{\command{git add}}
\newcommand{\gitcommit}{\command{git commit}}
\newcommand{\gitpush}{\command{git push}}
\newcommand{\gitpull}{\command{git pull}}

% Use this when displaying a new command

\newcommand{\command}[1]{``\lstinline{#1}''}
\newcommand{\program}[1]{\lstinline{#1}}
\newcommand{\url}[1]{\lstinline{#1}}
\newcommand{\channel}[1]{\lstinline{#1}}
\newcommand{\option}[1]{``{#1}''}
\newcommand{\step}[1]{``{#1}''}

\usepackage{pifont}
\newcommand{\checkmark}{\ding{51}}
\newcommand{\naughtmark}{\ding{55}}

\usepackage{listings}
\lstset{
  basicstyle=\small\ttfamily,
  columns=flexible,
  breaklines=true
}

\usepackage{fancyhdr}

\usepackage[margin=1in]{geometry}
\usepackage{fancyhdr}

\pagestyle{fancy}

\fancyhf{}
\rhead{Computer Science 280}
\lhead{Laboratory Assignment \assignmentnumber{}}
\rfoot{Page \thepage}
\lfoot{\duedate}

\usepackage{titlesec}
\titlespacing\section{0pt}{6pt plus 4pt minus 2pt}{4pt plus 2pt minus 2pt}

\newcommand{\labtitle}[1]
{
  \begin{center}
    \begin{center}
      \bf
      CMPSC 280\\Introduction to Computer Science I\\
      Fall 2017\\
      \medskip
    \end{center}
    \bf
    #1
  \end{center}
}

\begin{document}

\thispagestyle{empty}

\labtitle{Laboratory \assignmentnumber{} \\ \assigneddate{} \\ \duedate{}}

\section*{Objectives}

To continue to practice the use of GitHub to access the files for a laboratory assignment. Next, you will continue to
practice understanding a Python program's source code, revising a Markdown document, and running a test suite. As in a
previous assignment, you will also interact with a simulated customer, performing the tasks of verification and
validation, giving demonstrations, and finding and fixing defect(s) in the specification and/or source code of a
software application. However, the key focus of this assignment is project management and team work. Using GitHub and
the ``GitHub flow'' model, you will work in an assigned team to specify, design, implement, test, document, and release
the same list generation system from a previous laboratory assignment.

\section*{Suggestions for Success}

\begin{itemize}
  \setlength{\itemsep}{0pt}

\item {\bf Use the laboratory computers}. The computers in this laboratory feature specialized software for completing
  this course's laboratory  assignments. If it is necessary for you to work on a different machine, be sure to regularly
  transfer your work to a laboratory machine so that you can check its correctness. If you cannot use a laboratory
  computer and you need help with the configuration of your own laptop, then please carefully explain its setup to the
  systems administrator or the course instructor when you are asking questions.

\item {\bf Follow each step carefully}. Slowly read each sentence in the assignment sheet, making sure that you
  precisely follow each instruction. Take notes about each step that you attempt, recording your questions and ideas and
  the challenges that you faced. If you are stuck, then please tell the course instructor what assignment step you
  most recently completed.

\item {\bf Regularly ask and answer questions}. Please log into Slack at the start of a laboratory session and then join
  the appropriate channel. If you have a question about one of the steps in an assignment, then you can post it to the
  designated channel. Or, you can ask one of your team members or talk with course instructor in person or through
  Slack.

\item {\bf Store all your deliverables in GitHub}. As in past laboratory assignments, you will be responsible for
  storing all of your files (e.g., Python source code and Markdown-based writing) in a Git repository using GitHub and
  GitHub Classroom. Please verify that you have saved your source code in your Git repository by using \command{git
  status} to ensure that everything is updated. You can see if your assignment submission meets the established
  correctness requirements by using the provided checking tools on your local computer and in checking the commits in
  GitHub. Additionally, your team should perform all project management and communication tasks through GitHub. That is,
  all discussions about your programming systems product and the source code of and documentation for it must be
  available in GitHub.

\item {\bf Keep all of your files}. Don't delete your programs, output files, and written reports after you submit them
  through GitHub; you will need them again when you study for the quizzes and examinations and work on the other
  laboratory and final project assignments.

\item {\bf Back up your files regularly}. All of your files are regularly backed-up to the servers in the Department of
  Computer Science and, if you commit your files regularly, stored on GitHub. However, you may want to use a flash
  drive, Google Drive, or your favorite backup method to keep an extra copy of your files on reserve. In the event of
  any type of system failure, you are responsible for ensuring that you have access to a recent backup copy of all your
  files.

\item {\bf Hone your technical writing skills}. Computer science assignments require to you write technical
  documentation and descriptions of your experiences when completing each task. Take extra care to ensure that your
  writing is interesting and both grammatically and technically correct, remembering that computer scientists must
  effectively communicate and collaborate with their team members and the tutors, teaching assistants, and course
  instructor.

\item {\bf Review the Honor Code on the syllabus}. While you may discuss your assignments with others, copying source
  code or writing is a violation of Allegheny College's Honor Code.

\end{itemize}

\section*{Reading Assignment}

If you have not done so already, please read all of the relevant ``GitHub Guides'', available at
\url{https://guides.github.com/}, that explain how to use many of the features that GitHub provides. In particular,
please make sure that you have read guides such as ``Mastering Markdown'' and ``Documenting Your Projects on GitHub'';
each of them will help you to understand how to use both GitHub and GitHub Classroom. To do well on this assignment, you
should review Chapter 1 of the SETP textbook. You should also read Chapter 3 of SETP, paying particularly close
attention to the content about project management and personnel, work styles, and effort estimation. Finally, you and
your team members should read and discuss the content on the following web sites:

\vspace*{-.5em}

\begin{itemize}
  \setlength{\itemsep}{0pt}
  \item \url{https://guides.github.com/introduction/flow/}
  \item \url{https://help.github.com/articles/github-flow/}
  \item \url{https://help.github.com/categories/collaborating-with-issues-and-pull-requests/}
\end{itemize}

\vspace*{-.5em}

\section*{Organizing Your Software Development Team}

You should organize yourselves into the teams specified in \channel{\#announcements} channel in our Slack team. After
reviewing the reading assignments mention in the previous section and discussing the software that you are invited to
implement for this assignment, your team should discuss who will complete this project's tasks. As you are making task
assignments, please think about the strengths and weaknesses of the members of your development team. When it seems as
though there are no team members who best fit certain roles, you should make compromises to ensure that all work will
still be successfully finished. Your assignment of team members to roles should ensure that individuals have the
opportunity to explore new roles and learn more about the phases of the software engineering lifecycle. You should
ensure that all of your members know how to effectively use GitHub and then make a plan for how you will control your
source code and documentation and communication using GitHub features like the issue tracker, commit log, and pull
requests.

\section*{Accessing the Laboratory Assignment on GitHub}

To access the laboratory assignment, you should go into the \channel{\#announcements} channel in our Slack team and find
the announcement that provides a link for it. Now, make sure that the leader of your team also notes your team number
and first copies this link and pastes it into their web browser. Next, the team leader will create their team with the
name \command{Computer-Science-280-Fall-2017-Lab-3-Team-<assigned team number>} and then accept the laboratory
assignment and see that GitHub Classroom created a new GitHub repository for your team to access the assignment's
starting materials and to store the completed version of your assignment. At this point, each additional member of the
team can accept the assignment through GitHub. Please make sure that each of your team members joins the team to which
the instructor assigned them to work. Unless you provide the instructor with documentation of the extenuating
circumstances that you are facing, not accepting the assignment means that you automatically receive a failing grade for
it.

Now you are ready to download the starting materials to your laboratory computer. Click the ``Clone or download'' button
and, after ensuring that you have selected ``Clone with SSH'', please copy this command to your clipboard. To enter into
your course directory you should now type \command{cd cs280F2017}. By typing \command{git clone} in your terminal and
then pasting in the string that you copied from the GitHub site you will download all of the code for this assignment.
Before starting the assignment, make sure that each of your team members has the same starting materials.

Now, please study the \reflection{} in the repository. By reading this file you will learn how to run the \mainprogram{}
program and observe that its output is not correct. The \reflection{} will also teach you how to run the provided test
suite for this program and see that it fails. Please make sure that you carefully record your experiences when
completing each of these and other steps and that you understand the purpose, input, and output of each command. Now,
you will need to learn more about the Python programming language so that you can understand the requirements for this
program and ensure that it ultimately works correctly. If you need to modify the program's specification or
implementation, please carefully document your fix, making sure that the program produces the correct output and that
all of the test cases pass. Finally, you should add at least four new test cases that establish a confidence in the
correctness of your implementation.

\section*{Collaborating With Your Software Development Team}

Your team should use the GitHub issue tracker to write out a description of each of the key tasks that you need to
complete for this project. For now, the focus for these issues should be the features that you want your program to
have; later you can also use the issue tracker to discuss shortcomings of different aspects of your system. Attempting
to manage risk and estimate the effort required for individual team members to resolve these issues, now you should
assign people to roles and tasks. While it is acceptable for you to have in-person discussions with your team members,
please remember that all important discussions and decisions must be documented in GitHub.

Since your each team member is an administrator for your team's GitHub repository, everyone can create ``branches''
allocated to, for instance, each of the program features that you want to implement or the documentation that you want
to write. Once a specific branch contains the finished version of its associated deliverable, a team member should
create a pull request for discussion. If the team judges that the pull request has all of the expected characteristics,
then it should be merged into the ``master'' branch of your repository. If the pull request is not accepted, then the
team member should improve it until it meets each team member's expectations. Your team should continue to use this
model, called ``GitHub flow'', to support the completion of all deliverables.

For this assignment, you team should also report how well its Python source code scores with the \command{pylint3}.
Whenever possible, you should resolve the issues raised by \command{pylint3} --- for instance, your source code files
should always be correctly indented and there should always be the needed number of blank lines between functions.
Finally, you should also check the Markdown files in your repository with both \command{mdl} and \command{proselint},
resolving the identified issues whenever possible.

\section*{Evaluating Your Submission}

The source code for the \mainprogram{} should now contain the correct implementation. Next, all of the source code in
your test suite and your \mainprogram{} should have thoughtful and clear comments. Also, all of your Markdown files
(e.g., the one with a reflection) must meet the standards described in the Markdown Syntax Guide
\url{https://guides.github.com/features/mastering-markdown/}. Finally, all of your Markdown files should also adhere to
the Markdown standards established by the \step{Markdown linting} tool available at
\url{https://github.com/markdownlint/markdownlint/}

After the course instructor enables \step{continuous integration} with a system called Travis CI, when you use the
\gitpush{} command to transfer your source code to your GitHub repository, Travis CI will initialize a \step{build} of
your assignment, checking to see if it meets all of the requirements. If both your source code and writing meet all of
the established requirements, then you will see a green \checkmark{} in the listing of commits in GitHub after awhile.
If your submission does not meet the requirements, a red \naughtmark{} will appear instead. The instructor will reduce a
student's grade for this assignment if the red \naughtmark{} appears on the last commit in GitHub immediately before the
assignment's due date. Yet, if the green \checkmark{} appears on the last commit in your GitHub repository, then you
satisfied all of the main checks, thereby allowing the course instructor to evaluate other aspects of your source code
and writing, as further described in this assignment sheet. Unless you provide the course instructor with documentation
of the extenuating circumstances that you are facing, no late work will be considered towards your grade for this
laboratory assignment.

\section*{Summary of the Required Deliverables}

\noindent Students do not need to submit printed source code or technical writing for any assignment in this course.
Instead, this assignment invites you to submit, using GitHub, the following deliverables.

\begin{enumerate}

\setlength{\itemsep}{0in}

\item Placed at the start of \reflection{}, a high-level overview of the \mainprogramsource{}.

\item Placed near the start of \reflection{}, the requirements and design documents that clearly explain the inputs,
  outputs, and behaviors of the \mainprogramsource{} and how its main components will be organized into a
  well-documented, well-tested, and cohesive system.

\item Stored in \reflection{}, the user documentation that clearly explain how to run the \mainprogramsource{} program
  at the command line. In addition to highlighting how well your system adheres to a standard for Python programming,
  this documentation should also explain how to run the tests and detail the execution environment on which it was
  developed and tested.

\item Stored in \reflection{}, a planning document that outlines how different team members should complete deliverables
  that contribute to completion of this laboratory assignment.

\item A properly documented, well-formatted, and correct version of \mainprogramsource{} that both meets all of the
  established requirements and produces the desired output.

\item A properly documented, well-formatted, and correct version of \maintestsource{} that both meets all of the
  established requirements and passes correctly.

\item Placed at the end of \reflection{}, a two-paragraph reflection on the challenges that your team faced and the
  steps that you took to resolve them. This segment of the \reflection{} should also explain ways in which your team
  could enhance its project management strategies.

\end{enumerate}

\section*{Evaluation of Your Laboratory Assignment}

Using a report that the instructor shares with you through the commit log in GitHub, you will privately received a grade
on this assignment and feedback on your submitted deliverables. Your grade for the assignment will be a function of the
whether or not it was submitted in a timely fashion and if your program received a green \checkmark{} indicating that it
met all of the requirements. Other factors will also influence your final grade on the assignment. In addition to
studying the efficiency and effectiveness of your Python source code, the instructor will also evaluate the accuracy of
both your technical writing and the comments in your source code. If your submission receives a red \naughtmark{}, the
instructor will reduce your grade for the assignment while still considering the regularity with which you committed to
your repository and the overall quality of your partially completed work.

% Please see the instructor if you have questions about the evaluation of this laboratory assignment.

% \section*{Adhering to the Honor Code}

% In adherence to the Honor Code, students should complete this assignment on an individual basis. While it is appropriate
% for students in this class to have high-level conversations about the assignment, it is necessary to distinguish
% carefully between the student who discusses the principles underlying a problem with others and the student who produces
% assignments that are identical to, or merely variations on, someone else's work. Deliverables (e.g., Python source code
% or Markdown-based technical writing) that are nearly identical to the work of others will be taken as evidence of
% violating the \mbox{Honor Code}. Please see the course instructor if you have questions about this policy.

\end{document}
