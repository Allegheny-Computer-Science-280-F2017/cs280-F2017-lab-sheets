\documentclass[11pt]{article}

% NOTE: The "Edit" sections are changed for each assignment

% Edit these commands for each assignment

\newcommand{\assignmentduedate}{October 23}
\newcommand{\assignmentassignedate}{October 13}
\newcommand{\assignmentnumber}{One}

\newcommand{\labyear}{2017}
\newcommand{\assignedday}{Wednesday}
\newcommand{\dueday}{Friday}
\newcommand{\labtime}{9:00 am}

\newcommand{\assigneddate}{Announced: \assignedday, \assignmentassignedate, \labyear{} at \labtime{}}
\newcommand{\duedate}{Exam: \dueday, \assignmentduedate, \labyear{} at \labtime{}}

% Edit these commands to give the name to the main program

\newcommand{\mainprogram}{\lstinline{DisplayOutput}}
\newcommand{\mainprogramsource}{\lstinline{src/main/java/labone/DisplayOutput.java}}

% Edit this commands to describe key deliverables

\newcommand{\reflection}{\lstinline{writing/reflection.md}}

% Commands to describe key development tasks

% --> Running gatorgrader.sh
\newcommand{\gatorgraderstart}{\command{./gatorgrader.sh --start}}
\newcommand{\gatorgradercheck}{\command{./gatorgrader.sh --check}}

% --> Compiling and running program with gradle
\newcommand{\gradlebuild}{\command{gradle build}}
\newcommand{\gradlerun}{\command{gradle run}}

% Commands to describe key git tasks

% NOTE: Could be improved, problems due to nesting

\newcommand{\gitcommitfile}[1]{\command{git commit #1}}
\newcommand{\gitaddfile}[1]{\command{git add #1}}

\newcommand{\gitadd}{\command{git add}}
\newcommand{\gitcommit}{\command{git commit}}
\newcommand{\gitpush}{\command{git push}}
\newcommand{\gitpull}{\command{git pull}}

\newcommand{\gitcommitmainprogram}{\command{git commit src/main/java/labone/DisplayOutput.java -m "Your
descriptive commit message"}}

% Use this when displaying a new command

\newcommand{\command}[1]{``\lstinline{#1}''}
\newcommand{\program}[1]{\lstinline{#1}}
\newcommand{\url}[1]{\lstinline{#1}}
\newcommand{\channel}[1]{\lstinline{#1}}
\newcommand{\option}[1]{``{#1}''}
\newcommand{\step}[1]{``{#1}''}

\usepackage{pifont}
\newcommand{\checkmark}{\ding{51}}
\newcommand{\naughtmark}{\ding{55}}

\usepackage{listings}
\lstset{
  basicstyle=\small\ttfamily,
  columns=flexible,
  breaklines=true
}

\usepackage{fancyhdr}

\usepackage[margin=1in]{geometry}
\usepackage{fancyhdr}

\pagestyle{fancy}

\fancyhf{}
\rhead{Computer Science 111}
\lhead{Exam \assignmentnumber{}}
\rfoot{Page \thepage}
\lfoot{\duedate}

\usepackage{titlesec}
\titlespacing\section{0pt}{6pt plus 4pt minus 2pt}{4pt plus 2pt minus 2pt}

\newcommand{\guidetitle}[1]
{
  \begin{center}
    \begin{center}
      \bf
      CMPSC 280\\Software Engineering\\
      Fall 2017\\
      \medskip
    \end{center}
    \bf
    #1
  \end{center}
}

\begin{document}

\thispagestyle{empty}

\guidetitle{Exam \assignmentnumber{} Study Guide \\ \assigneddate{} \\ \duedate{}}

\section*{Introduction}

\noindent
The quiz will be ``closed notes'' and ``closed book'' and it will cover the
following materials. Please review the ``Course Schedule'' on the Web site for
the course to see the content and slides that we have covered to this date.
Students may post questions about this material to our Slack team.

\begin{itemize}

  \itemsep 0in

  \item Chapters One through Six SETP by Pfleeger and Atlee.

  \item Chapters One through Four in MMM by Frederick P.\ Brooks.

  \item Implementing, testing and debugging of Python programs; writing
    technical documentation with Markdown, editing and executing Python programs
    in Linux; basic and advanced commands for using {\tt git} and GitHub;
    practical strategies for effective teamwork.

  \item Your class notes and lecture slides, labs 1--4, and the handouts from
    lab.

\end{itemize}

\noindent The examination will include a mix of questions that will require you
to draw and/or comment on a diagram, write a short answer, explain and/or write
a source code segment, or give and comment on a list of concepts or points. The
emphasis will be on the following list of illustrative subjects. Please note
that this list is not exhaustive --- rather it is designed to suggest
representative topics.

The course instructor encourages students to form and participate in study
groups when studying the content that this review sheet outlines. If students
have questions about the content on this review sheet, they should schedule a
meet with the instructor during office hours.

% \vspace*{-1em}

\subsubsection*{Detailed Review of the SETP Chapters}

\begin{enumerate}
  \itemsep 0in

  \item Chapter 1

    \begin{enumerate}
      \itemsep 0in
      \item Basic definitions of the term software engineering
      \item Terminology use to explain software defects and software quality
      \item Roles of members in a software engineering team
    \end{enumerate}

  \item Chapter 2

    \begin{enumerate}
      \itemsep 0in
      \item The phases of the software engineering lifecycle
      \item Tasks that are completed in the software engineering phases
      \item The goals of verification and validation during software engineering
    \end{enumerate}

  \item Chapter 3

    \begin{enumerate}
      \itemsep 0in
      \item Techniques for breaking down and scheduling software projects
      \item Ways to organize the members of a software engineering team
      \item Methods for estimating the effort needed to complete engineering
        tasks
    \end{enumerate}

  \item Chapter 4

    \begin{enumerate}
      \itemsep 0in
      \item The types and characteristics of software requirements
      \item Strategies for ensuring that software requirements are testable
      \item The ways in which diagrams can support requirement understanding
    \end{enumerate}

  \item Chapter 5

    \begin{enumerate}
      \itemsep 0in
      \item The similarities and differences between architecture and design
      \item Different views of the architecture and design of a software system
      \item Strategies for creating and documenting software architectures and designs
    \end{enumerate}

  \item Chapter 6

    \begin{enumerate}
      \itemsep 0in
      \item The characteristics of a software system's design (e.g., coupling
        and cohesion)
      \item How abstraction and information hiding support design and
        implementation
      \item The ways in which diagrams can support the design of a software
        system
    \end{enumerate}

\end{enumerate}

\vspace*{-1em}

\subsubsection*{Overview of the MMM Chapters}

\begin{enumerate}
  \item 
\end{enumerate}

\section*{Reminder Concerning the Honor Code}

\noindent Students are required to fully adhere to the Honor Code during the
completion of this quiz. More details about the Allegheny College Honor Code are
provided on the syllabus. Students are strongly encouraged to carefully review
the full statement of the Honor Code before taking this quiz. If you do not
understand Allegheny College's Honor Code, please schedule a meeting with the
course instructor. The following is a review of Honor Code statement from the
course syllabus:

\begin{quote}

  The Academic Honor Program that governs the entire academic program at
  Allegheny College is described in the Allegheny Academic Bulletin. The Honor
  Program applies to all work that is submitted for academic credit or to meet
  non-credit requirements for graduation at Allegheny College. This includes all
  work assigned for this class (e.g., examinations, laboratory assignments, and
  the final project). All students who have enrolled in the College will work
  under the Honor Program. Each student who has matriculated at the College has
  acknowledged the following pledge:

\end{quote}

\begin{quote}

  I hereby recognize and pledge to fulfill my responsibilities, as defined in the Honor Code, and to maintain the
  integrity of both myself and the College community as a whole.

\end{quote}

\end{document}
