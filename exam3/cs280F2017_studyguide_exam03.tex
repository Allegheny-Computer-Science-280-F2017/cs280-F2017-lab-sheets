\documentclass[11pt]{article}

% NOTE: The "Edit" sections are changed for each assignment

% Edit these commands for each assignment

\newcommand{\assignmentduedate}{December 15}
\newcommand{\assignmentassignedate}{December 11}
\newcommand{\assignmentnumber}{Three}

\newcommand{\labyear}{2017}
\newcommand{\assignedday}{Monday}
\newcommand{\dueday}{Friday}
\newcommand{\labtime}{7:00 pm}
\newcommand{\announcetime}{1:30 pm}

\newcommand{\assigneddate}{Announced: \assignedday, \assignmentassignedate, \labyear{} at \announcetime{}}
\newcommand{\duedate}{Exam: \dueday, \assignmentduedate, \labyear{} at \labtime{}}

% Edit these commands to give the name to the main program

\newcommand{\mainprogram}{\lstinline{DisplayOutput}}
\newcommand{\mainprogramsource}{\lstinline{src/main/java/labone/DisplayOutput.java}}

% Edit this commands to describe key deliverables

\newcommand{\reflection}{\lstinline{writing/reflection.md}}

% Commands to describe key development tasks

% --> Running gatorgrader.sh
\newcommand{\gatorgraderstart}{\command{./gatorgrader.sh --start}}
\newcommand{\gatorgradercheck}{\command{./gatorgrader.sh --check}}

% --> Compiling and running program with gradle
\newcommand{\gradlebuild}{\command{gradle build}}
\newcommand{\gradlerun}{\command{gradle run}}

% Commands to describe key git tasks

% NOTE: Could be improved, problems due to nesting

\newcommand{\gitcommitfile}[1]{\command{git commit #1}}
\newcommand{\gitaddfile}[1]{\command{git add #1}}

\newcommand{\gitadd}{\command{git add}}
\newcommand{\gitcommit}{\command{git commit}}
\newcommand{\gitpush}{\command{git push}}
\newcommand{\gitpull}{\command{git pull}}

\newcommand{\gitcommitmainprogram}{\command{git commit src/main/java/labone/DisplayOutput.java -m "Your
descriptive commit message"}}

% Use this when displaying a new command

\newcommand{\command}[1]{``\lstinline{#1}''}
\newcommand{\program}[1]{\lstinline{#1}}
\newcommand{\url}[1]{\lstinline{#1}}
\newcommand{\channel}[1]{\lstinline{#1}}
\newcommand{\option}[1]{``{#1}''}
\newcommand{\step}[1]{``{#1}''}

\usepackage{pifont}
\newcommand{\checkmark}{\ding{51}}
\newcommand{\naughtmark}{\ding{55}}

\usepackage{listings}
\lstset{
  basicstyle=\small\ttfamily,
  columns=flexible,
  breaklines=true
}

\usepackage{fancyhdr}

\usepackage[margin=1in]{geometry}
\usepackage{fancyhdr}

\pagestyle{fancy}

\fancyhf{}
\rhead{Computer Science 111}
\lhead{Exam \assignmentnumber{}}
\rfoot{Page \thepage}
\lfoot{\duedate}

\usepackage{titlesec}
\titlespacing\section{0pt}{6pt plus 4pt minus 2pt}{4pt plus 2pt minus 2pt}

\newcommand{\guidetitle}[1]
{
  \begin{center}
    \begin{center}
      \bf
      CMPSC 280\\Software Engineering\\
      Fall 2017\\
      \medskip
    \end{center}
    \bf
    #1
  \end{center}
}

\begin{document}

\thispagestyle{empty}

\guidetitle{Exam \assignmentnumber{} Study Guide \\ \assigneddate{} \\ \duedate{}}

\section*{Introduction}

\noindent
The exam will be ``closed notes'' and ``closed book'' and it will cover the
following materials. Please review the ``Course Schedule'' on the Web site for
the course to see the content and slides that we have covered to this date.
Students may post questions about this material to our Slack team.

\begin{itemize}

  % \itemsep 0in

  \item Chapters One through Three in SETP (i.e., introduction to the software
    engineering lifecycle)

  \item Chapters One through Three in MMM (i.e., challenges and solutions in
    software engineering)

  \item Chapters Four through Six in SETP (i.e., requirements, architecture, and design)

  \item Chapter Four in MMM (i.e., trade-offs in strategies for system design)

  \item Chapter Seven in SETP (i.e., methods for software implementation)

  \item Chapters Twelve and Thirteen in MMM (i.e., tools for implementation and
    debugging)

  \item Chapter Eight in SETP (i.e., program testing techniques)

  \item Chapter Nine in SETP (i.e., principles of system testing)

  \item Chapters Sixteen in MMM (i.e., the ``no silver bullet'' hypothesis)

  \item Chapter Ten in SETP (i.e., strategies for system delivery)

  \item Chapter Eleven in SETP (i.e., software maintenance techniques)

  \item Your class notes, class activities, lecture slides, and the first four
    laboratory assignments

  \item Knowledge of the basic commands necessary for using {\tt git} and
    GitHub; understanding of the ``Git flow'' model for collaboration; basic
    understanding of the Markdown syntax

  \item Key concepts introduced and reviewed during the laboratory sessions
    (e.g., the components of a ``programming systems product'' and how to use
    testing and linting tools in Python)

  \item Lessons learned from the team-based implementation and delivery of
    real-world software

\end{itemize}

\noindent The examination will include a mix of questions that will require you
to draw and/or comment on a diagram, write a short answer, explain and/or write
a source code segment, or give and comment on a list of concepts or points. The
emphasis will be on the following list of illustrative subjects. Please note
that this list is not exhaustive --- rather it is designed to suggest
representative topics.

When studying for the test, don't forget that the Web site for our course
contains mobile-ready slides that will provide you with an overview of the key
concepts that we discussed in the first modules. You can use the color scheme in
the slides to notice points where we, for instance, completed an in-class
activity, discussed a key point, or made reference to additional details
available in the one of the textbooks. The course instructor encourages students
to form and participate in study groups when studying the content that this
review sheet outlines. If students have questions about the material on this
review sheet, they should schedule a meeting with the instructor.

\vspace*{-.05in}
\begin{itemize}

  \item The state-of-the-art and the key challenges within the field of software
    engineering, with a focus on the steps of problem solving and the meaning of
    terms like ``defect'' and ``quality''.

  \item The phases of the software development lifecycle and the ways in which
    different software process models (e.g., the spiral model or the V model)
    connect and interpret these phases.

  \item The key strengths and weaknesses of the different software development
    process models (e.g., one drawback of the waterfall model is its focus on
    documents and its lack of explicit iteration).

  \item Key terms such as ``verification'' and ``validation'' and
    ``incremental'' and ``iterative''.

  \item The role that software testing plays in different phases of the software
    engineering lifecycle.

  \item How to use activity graphs or task lists to track progress and plan a
    software development project. Additionally, an understanding of the ways in
    which managers will estimate the deadlines for completing a software system
    (e.g., using data mining algorithms to predict project characteristics such
    as anticipated costs and the likelihood of an on-time completion).

  \item The roles that members of a software team may play and the ways in which
    individual personalities and characteristics may equip certain people to
    work on specific tasks.

  \item How different types of software engineering tasks exhibit different
    relationships between the time-to-completion and the number of workers
    assigned to finish the task.

  \item Lessons learned from working in a team to specify, design, implement,
    test, document, and release a programming systems product during our
    laboratory sessions. For instance, an understanding of the technical means
    by which a program's code and documentation is released to GitHub. In
    addition, a grasp of how to use GitHub's issue tracking system to report,
    triage, and resolve defects and/or concerns about a programming systems
    product.

  \item The reasons why it is important to accurately capture the requirements
    for a software system.

  \item The different ways in which you can classify and understand software
    requirements.

  \item Characteristics of requirements and the ways in which they can be
    validated and verified.

  \item The relationship between the architecture and design of a software
    system.

  \item The characteristics of a software architecture and how they may
    influence the performance (e.g., response time, throughput, and load)
    demonstrated by the completed system.

  \item Canonical software architectures (e.g., ``pipe and filter'') and their
    strengths and weaknesses.

  \item The overall meaning and purpose of object-oriented design and
    implementation measures.

  \item The principles for improving and evaluating a system's design. For
    instance, in the context of coupling, an understanding of tight and loose
    coupling, the kinds of architectures that promote different types of
    coupling, and an intuitive understanding of coupling scores.

  \item The ways in which pre-conditions, post-conditions, and invariants can
    document an interface.

  \item A basic understanding of object-oriented design and the effective use of
    design patterns.

  \item The different types of Unified Modelling Language (UML) diagrams, the
    purpose, strengths, and weaknesses of each canonical type of diagram, and,
    overall, the purpose of using diagrams in software design. For instance, an
    understanding of the UML package and class diagrams.

  \item An understanding of the phrase ``conceptual integrity'' and,
    additionally, the ability to analyze different strategies for achieving
    conceptual integrity in all phases of the software lifecycle.

  \item The tension between ``aristocracy'' and ``democracy'' in the design of
    software systems.

  \item General guidelines for improving the implementation of a software
    system. For instance, the ways in which a software development team can
    achieve objectives such as ``localizing input and output'', ``including
    psuedo-code'', and ``revising and rewriting, not patching''.

  \item Standards and guidelines that govern the way in which a program's source
    code should be implemented and formatted. A high-level understanding of a
    source code standard for the Python and Java programming language and a
    clear sense of the benefits and drawbacks associated with using this
    standard. Additionally, the ability to look at a source code segment,
    written in Python or Java, and comment on how it adheres to or violates a
    coding standard.

  \item Standards and guidelines for creating internal and external
    documentation for a software system. Given, for instance, a source code
    comment, you should be able to explain its strengths and weaknesses. You
    should also understand how external documentation should describe the
    problem, algorithms, and data that are part of a system's implementation.

  \item An understanding of how programming is a form of problem solving.
    Specifically, you should understand the activities associated with
    understanding a problem, devising a plan, carrying out the plan, and
    reflecting on the chosen implementation of the completed software system.

  \item The ways in which agile methodologies, such as extreme programming and
    pair programming, influence the efficiency of software implementation and
    the quality of the resulting product.

\end{itemize}

\section*{Reminder Concerning the Honor Code}

\noindent Students are required to fully adhere to the Honor Code during the
completion of this exam. More details about the Allegheny College Honor Code are
provided on the syllabus. Students are strongly encouraged to carefully review
the full statement of the Honor Code before taking this exam. If you do not
understand Allegheny College's Honor Code, please schedule a meeting with the
course instructor. The following is a review of Honor Code statement from the
course syllabus:

\begin{quote}

  The Academic Honor Program that governs the entire academic program at
  Allegheny College is described in the Allegheny Academic Bulletin. The Honor
  Program applies to all work that is submitted for academic credit or to meet
  non-credit requirements for graduation at Allegheny College. This includes all
  work assigned for this class (e.g., examinations, laboratory assignments, and
  the final project). All students who have enrolled in the College will work
  under the Honor Program. Each student who has matriculated at the College has
  acknowledged the following pledge:

\end{quote}

\begin{quote}

  I hereby recognize and pledge to fulfill my responsibilities, as defined in
  the Honor Code, and to maintain the integrity of both myself and the College
  community as a whole.

\end{quote}

\section*{Strategies for Studying}

As you study for this examination, you are encouraged to form study groups with
individuals who were previously a member of one of your software development
teams during a laboratory session. You can collaborate with these individuals to
ensure that you understand all of the key concepts mentioned on this study
guide. Additionally, students are encouraged to create a Slack channel that can
host questions and answers that arise as you are studying for the test. Even
though the course instructor will try to, whenever possible, answer review
questions that students post in this channel, you are strongly encouraged to
answer the questions posted by your colleagues as this will also help you to
ensure that you fully understand the material.

When studying for the test, don't forget that the Web site for our course
contains mobile-ready slides that will provide you with an overview of the key
concepts that we discussed in the first module. You can use the color scheme in
the slides to notice points where we, for instance, completed an in-class
activity, discussed a key point, or made reference to additional details
available in the SETP and MMM textbooks. Finally, students are strongly
encouraged to schedule a meeting during the course instructor's office hours so
that we can resolve any of your questions about the material and ensure that you
have the knowledge and skills necessary for doing well on this examination.

Remember, while the test is taken individually during the stated class period,
your review for it can be done collaboratively! Again, you can schedule a
meeting with the course instructor during office hours if you have questions
about the topics or policies outlined on this review sheet.

\end{document}
