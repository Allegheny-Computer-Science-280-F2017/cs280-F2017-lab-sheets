\documentclass[11pt]{article}

% NOTE: The "Edit" sections are changed for each assignment

% Edit these commands for each assignment

\newcommand{\assignmentduedate}{December 12}
\newcommand{\assignmentassignedate}{November 14}
\newcommand{\assignmentnumber}{Seven}

\newcommand{\labyear}{2017}
\newcommand{\labday}{Tuesday}
\newcommand{\labtime}{2:30 pm}

\newcommand{\assigneddate}{Assigned: \labday, \assignmentassignedate, \labyear{} at \labtime{}}
\newcommand{\duedate}{Due: \labday, \assignmentduedate, \labyear{} at \labtime{}}

% Edit these commands to give the name to the main program

\newcommand{\mainprogram}{\lstinline{generator.py}}
\newcommand{\mainprogramsource}{\lstinline{generator/generator.py}}
\newcommand{\maintestsource}{\lstinline{generator/tests/test_generator.py}}

% Edit this commands to describe key deliverables

\newcommand{\reflection}{\lstinline{README.md}}

% Commands to describe key git tasks

\newcommand{\gitcommitfile}[1]{\command{git commit #1}}
\newcommand{\gitaddfile}[1]{\command{git add #1}}

\newcommand{\gitadd}{\command{git add}}
\newcommand{\gitcommit}{\command{git commit}}
\newcommand{\gitpush}{\command{git push}}
\newcommand{\gitpull}{\command{git pull}}

% Command for writing authors of a paper

\usepackage{xspace}
\newcommand{\etal}{et al.\xspace}

% Use this when displaying a new command

\newcommand{\command}[1]{``\lstinline{#1}''}
\newcommand{\program}[1]{\lstinline{#1}}
\newcommand{\url}[1]{\lstinline{#1}}
\newcommand{\channel}[1]{\lstinline{#1}}
\newcommand{\option}[1]{``{#1}''}
\newcommand{\step}[1]{``{#1}''}

\usepackage{pifont}
\newcommand{\checkmark}{\ding{51}}
\newcommand{\naughtmark}{\ding{55}}

\usepackage{listings}
\lstset{
  basicstyle=\small\ttfamily,
  columns=flexible,
  breaklines=true
}

\usepackage{fancyhdr}

\usepackage[margin=1in]{geometry}
\usepackage{fancyhdr}

\pagestyle{fancy}

\fancyhf{}
\rhead{Computer Science 280}
\lhead{Final Project}
\rfoot{Page \thepage}
\lfoot{\duedate}

\usepackage{titlesec}
\titlespacing\section{0pt}{6pt plus 4pt minus 2pt}{4pt plus 2pt minus 2pt}

\newcommand{\labtitle}[1]
{
  \begin{center}
    \begin{center}
      \bf
      CMPSC 280\\Software Engineering\\
      Fall 2017\\
      \medskip
    \end{center}
    \bf
    #1
  \end{center}
}

\begin{document}

\thispagestyle{empty}

\labtitle{Final Project \\ \assigneddate{} \\ \duedate{}}

\section*{Objectives}

% The goal of this assignment is to investigate programming as a problem solving task, focusing on the necessity of
% reflecting on the understanding on the problem, the plan chosen to solve the problem, and the execution of this plan.
% You will be responsible for using each of the software tools that you created in the last two laboratory assignments,
% identifying areas in which they can be improved.

Building on the knowledge and skills that you acquired in the previous laboratory assignments, in this final project,
you will specify, design, implement, test, document, and maintain a complete programming systems product. Specifically,
the goal of this final project is to ensure that you have the opportunity to explore and master all of the activities of
every phase of the software engineering lifecycle. The ultimate software product resulting from your collaborative work
on this final project is a useful, yet complex, system that solves an important problem. Specifically, you will create a
system, called GatorGauge, that automatically analyzes the GitHub repositories that contain laboratory solutions
submitted by students in an introductory computer science course. GatorGauge will then present insights about how these
novice programmers completed their assignments, surfacing relevant points about how they, for instance, programmed in
Java and used GitHub. As in previous assignments, you will continue to apply effective project management and team work
strategies to create an open-source project that is available on GitHub. Using the ``GitHub flow'' model, you will
improve and evaluate these projects and the processes that governed their creation.

\section*{Reviewing the Prior Programming Systems Products}

At the start of the final project, it is useful for you and your classmates to review the completed laboratory
assignments and the three programming systems products that are outlined in the following descriptions. As you review
each of these assignments and the three software products, please reflect on the tasks that you found to be the most
challenging and the strategies that you developed for addressing both technical and managerial problems. Please think
carefully about the Python language features and packages that you found to be the most useful in completing the prior
laboratory assignments. Next, consider the best ways in which your class can use Travis, GitHub, and other external
tools (e.g., Coveralls) to complete this final project. Finally, ask yourself two reflective questions. If you could
start over these laboratory assignments, what would you do differently? What do you think that you and your team members
did well on the past assignments?

\subsubsection*{GatorGrouper: A Software Tool for Forming Teams of Programmers}

It is common for projects in computer science courses to involve teamwork. Since there is evidence that there are
downsides associated with allowing students to form their own teams, course instructors often implement bespoke software
tools to automatically creates these teams. However, these programs are not general-purpose, often requiring revision each
type an instructor teaches a new class. Moreover, these tools often resort to randomly forming groups, ignoring the
characteristics of the students in the course. These are the problems that GatorGrouper solves! Given access to
information about the students in a course, GatorGrouper automatically forms teams of programmers, using either random
algorithms or methods that strategically devise the groups.

\subsubsection*{Edurate: A Software Tool for Rating Software Educators}

While instructors at many colleges and universities are evaluated by students through student surveys at the end of a
semester, most institutions do not have a system for intermediate evaluations. Even if an instructor were to implement a
Google Form and store the results of surveys in Google Drive, these results are not yet fully analyzed by the supporting
software. This is the problem that Edurate solves! Given access to a spreadsheet of results from in-class survey ratings
completed by students, Edurate analyzes the numerical and textual data to give course instructors concrete and
actionable ideas for how to improve their teaching. Along with pointing out key numerical trends, Edurate uses natural
language processing methods, like those implemented in the Gensim Python library, to extract and present keywords and
topics about the instructor's teaching.

\subsubsection*{Accelergator: A Software Tool for Accelerated and Adaptive Advising}

Even though instructors at many colleges and universities are expected to provide academic advising to students, most
institutions do not provide any software to support this critical task. Even if an instructor were to implement a Google
Form to collect data and status updates from their advisees and store the results from these forms in Google Drive, they
would not be fully analyzed by the supporting software. This is the problem that Accelergator solves! Given access to a
spreadsheet of data points about the academic interests and standing of advisees, Accelergator analyzes the numerical
and textual responses to give advisors concrete and actionable ideas for how to adaptively advise their students. Along
with pointing out key numerical trends and checkpoint completion status, Accelergator uses natural language processing
methods, like those implemented in the Gensim library, to extract and present keywords and topics about an instructor's
advisees.

\noindent
Students can access the current version of these three software tools by visiting these web sites:

\begin{enumerate}
  \item \url{https://github.com/GatorGrouper/gatorgrouper}
  \item \url{https://github.com/Edurate/edurate}
  \item \url{https://github.com/Accelegator/accelegator}
\end{enumerate}

\noindent Students who have questions about any of these three programming systems products and how they related to the
current final project are encouraged to speak to the course instructor during office hours. Please note that if you
reuse source code from one of the aforementioned GitHub repositories, you must properly document this reuse in the code
itself and in its documentation.

\section*{GatorGauge: A Tool for Characterizing Programs at Scale}

In a paper entitled ``HappyFace: Identifying and Predicting Frustrating Obstacles for Learning Programming at Scale'',
Drosos~\etal{} describe a software tool, called HappyFace, that allows programmers to report when they are frustrated
during the completion of a programming exercise. For this final project, you will implement GatorGauge, a system
inspired by the aforementioned paper. Instead of requiring an enhanced integrated development environment, as is the
case for the HappyFace system and its reliance on PythonTutor, GatorGauge will accept as input a directory that contains
all of the GitHub repositories for every student and every laboratory assignment submission for an entire semester of an
introductory computer science course using GitHub Classroom. Working with the instructor, this directory should be
created by interaction with GitHub. While there are many tools that enable the downloading of all the GitHub
repositories in an organization, you are encouraged to explore the use of
\url{https://github.com/danwallach/github-clone-all}, which has been tested by the course instructor and determined to
be suitable for this first task.

GatorGauge should accept an input parameter that specifies the directory containing the subdirectories that are the
GitHub repositories for every student and laboratory assignment. Then, GatorGauge will traverse all of these
directories, using functions that enable it to, for instance, find all of the Java source code files or the natural
language reflections stored in Markdown files. Next, GatorGauge should provide the following features, to be confirmed
with your customer.

\begin{itemize}

  \item {\bf Source Code Comments}: The tool should extract, count, and analyze the source code comments in the Java
    programming langauge files. This feature will require that GatorGauge can differentiate between single-line and
    multiple-line comments and that it can, additionally, extract the natural language text of each comment. In addition
    to counting the comments and reporting relevant summary statistics concerning these counts across all students and
    all GitHub repositories, GatorGauge should apply natural language processing techniques extract both keywords and
    the commonly expressed topics in the source code comments. \\[.5em] \noindent {\em Suggested Tools\/}: The comment
    analysis functions in GatorGrader, available on GitHub.

  \item {\bf Java Source Code}: The tool should extract, count, and analyze the Java source code in every one of the
    GitHub repositories in the provided directory. This feature will require GatorGauge to parse Java source code and,
    for instance, extract details about the number of lines of source code and the number of declared variables in each
    of the Java source code files. In addition to counting these entities, GatorGauge should provide summary statistics.
    \\[.5em] \noindent {\em Suggested Resources\/}: Online tutorials about regular expressions in the Python language.

  \item {\bf GitHub Commits}: The tool should extract, count, and analyze the GitHub commits associated with every one
    of the GitHub repositories in the provided directory. This feature will require GatorGauge to have functionality
    that enables it to run \command{git} commands and process their output. Once the commit logs are accessible to
    GatorGauge, the tool should, for instance, count the number of commits, process the commit messages, and examine the
    changed code in each commit, surfacing interesting patterns. Leveraging ideas from the HappyFace system, future uses
    of the GatorGauge tool might invite students to include an affective emoji in each of their commits, thereby
    enabling the tool to assess the affective state of a programmers when they commit code. Additionally, GatorGauge can
    use natural language processing to extract the keywords, topics, and emotions associated with all of the commits. \\[.5em]
    \noindent {\em Suggested Tools\/}: The GitPython, dulwich, sh, and plumbum packages, available on GitHub.

  \item {\bf Technical Writing}: As the laboratory assignments in introductory computer science classes at Allegheny
    College require students to write technical reports and reflections, the GatorGauge tool should have features that
    can analyze the Markdown-based documents. Once these files are available to GatorGauge, it should use natural
    language processing to extract keywords, topics, and emotions associated with the writing for each of the laboratory
    assignments and across all of the students. This analysis should be customized to the writing prompts for each
    assignment: for instance, if one reflection is about technical challenges and another is about teamwork, then
    GatorGauge should account for these differences. \\[.5em] \noindent {\em Suggested Tools\/}: The Gensim, nltk, and
    TextBlob packages, available on GitHub.

\end{itemize}

Ultimately, the goal of the GatorGauge tool is to analyze and understand the GitHub repositories that novice programmers
create when they complete laboratory assignments. The understanding that GatorGauge automatically produces will serve
two critical roles. First, it will enable course instructors to understand, across a significant number of diverse
solution, the challenges that students face when completing laboratory assignments. This understanding would then allow
an instructor to revise the assignment so as to avoid pitfalls that students commonly faced. Secondly, a course
instructor can present the results from GatorGauge's analysis to the students either during a course or in a follow-on
semester of the course. This will then ensure that students are aware of the challenges were faced in completing an
assignment, thus ensuring that they can better manage their time and improving their awareness of key difficulties.
Right now, the many GitHub repositories that students create in introductory computer science classes remain largely
unanalyzed --- GatorGauge will solve this problem by surfacing interesting and actionable insights!

\section*{Accessing the Laboratory Assignment on GitHub}

To access the laboratory assignment, you should go into the \channel{\#announcements} channel in our Slack team and find
the announcement that provides a link for it. Copy this link and paste it into your web browser. Now, you should accept
the laboratory assignment and see that GitHub Classroom created a new GitHub repository for you to access the
assignment's review materials and to store the completed version of your review. Specifically, to access your new GitHub
repository for this assignment, please click the green ``Accept'' button and then click the link that is prefaced with
the label ``Your assignment has been created here''. If you accepted the assignment and correctly followed these steps,
you should have created a GitHub repository with a name like
``Allegheny-Computer-Science-280-Fall-2017/computer-science-280-fall-2017-lab-7-gkapfham''. Unless you provide the
instructor with documentation of the extenuating circumstances that you are facing, not accepting the assignment means
that you automatically receive a failing grade for it.

You will use the \reflection{} file in this GitHub repository to report on all of the work that you completed and review
the work that your class members finished. Specifically, you should use the \reflection{} file to evaluate the
effectiveness of the class, commenting on who completed what work and how well the work was completed. Make sure that
you review all of the questions in the previous section of the assignment as you write your \reflection{}. Finally,
considering the definitions given in MMM, you should also assess the quality of the programming systems product that the
class submitted before the deadline. Please see the instructor with any questions about this task.

As in past assignments, you can click the ``Clone or download'' button and, after ensuring that you have selected
``Clone with SSH'', please copy this command to your clipboard. To enter your course directory you should now type
\command{cd cs280F2017}. By typing \command{git clone} in your terminal and then pasting in the string that you copied
from the GitHub site you will download all of the code for this assignment. Before starting the assignment, make sure
that the other students in the class can access to their own private repository and everyone understands what type of
content they should include in the \reflection{} file. You may talk with the course instructor if you have questions
about the purpose of the GitHub repository that only contains a single \reflection{} file.

\section*{Summary of the Required Deliverables}

Using a report that the instructor shares with you through the commit log in GitHub, you will privately received a grade
on this assignment and feedback on your deliverables. The instructor will also evaluate the effectiveness with which the
class completed the assignment. Ultimately, this laboratory assignment invites you to submit, using GitHub, the
following deliverables.

\vspace*{-.5em}

\begin{enumerate}

\setlength{\itemsep}{0in}

\item Stored in a \reflection{} file, a reflection on your contributions to the three projects and the contributions
  made by the other members of this class. Refer to the relevant sections of this sheet for details about what to write
  in your repository's \reflection{} file. Notably, you must thoughtfully evaluate your work and the work of the other
  class members, including source code and documentation, for the GatorGauge project that is the focus of this
  assignment.

\item A properly documented, well-formatted, and correct version of all of the source code needed to fulfill the
  requirements given in the issue tracker of the repository for GatorGauge.

\item Stored in the \reflection{} file of GatorGauge's GitHub repository, user documentation that explains how to use
  all of the relevant features of your programming systems product.

\end{enumerate}

As previously mentioned, the instructor will also evaluate the effectiveness with which the class completed this
assignment. The instructor will assess team effectiveness as {\bf excellent} when:

\vspace*{-.5em}

\begin{enumerate}
  \setlength{\itemsep}{0pt}

  \item All team members regularly contribute, in a timely fashion, to the GitHub project through the issue tracker,
    pull requests, and the committing of source code and documentation.

  \item All team members contribute correct, useful Python source code and/or Markdown documentation that supports the
    creation of a working and valuable programming systems product.

  \item All team members participate in the classroom discussions, sharing status updates and ideas.

  \item All team members regularly communicate with others in a respectful and informative manner.

  \item All team members work with others and the instructor to resolve any differences that arise.

\end{enumerate}

\vspace*{-.5em}

The instructor will assess team effectiveness as {\bf good} when the majority of the team members fulfill the
aforementioned standards. The team's effectiveness will be assessed as {\bf average} when some, but not the majority, of
the team members make meaningful contributions to the project, according to these set standards. The course instructor
will assess team effectiveness as {\bf below average} when, as a whole, the team does not function according to the
above standards.

The instructor will assess product quality as {\bf excellent} when, by the established deadline, the products' GitHub
repository contains a full programming systems product that contains: clear and useful documentation, a comprehensive
and passing test suite that effectively covers the code base, and correct implementations of all, or the majority of,
the requirements listed in the issue tracker. The instructor will assess product quality as {\bf good} when all of the
aforementioned standards are met, but some of the requirements are not fully implemented or not implemented correctly.
The product's quality will be assessed as {\bf average} when some, but not the majority, of the requirements are
properly fulfilled. The course instructor will assess product quality as {\bf below average} when, as a whole, the
product does not feature new requirements or there is a regression in functionality.

The course instructor expects that each of the three projects will be fully completed and ready for use by instructors
and/or students in the Department of Computer Science. For the GatorGauge final project, every student will receive the
same baseline grade for team effectiveness and product quality. However, an individual student grade will be adjusted
higher or lower according to their contribution's to the project, as reflected in the material that they and others
include in the \reflection{} file of the private GitHub repository and in the public repository of each project's GitHub
repository. Please see the instructor if you have questions about the grading of this assignment.

\end{document}
