\documentclass[11pt]{article}

% NOTE: The "Edit" sections are changed for each assignment

% Edit these commands for each assignment

\newcommand{\assignmentduedate}{November 14}
\newcommand{\assignmentassignedate}{November 7}
\newcommand{\assignmentnumber}{Six}

\newcommand{\labyear}{2017}
\newcommand{\labday}{Tuesday}
\newcommand{\labtime}{2:30 pm}

\newcommand{\assigneddate}{Assigned: \labday, \assignmentassignedate, \labyear{} at \labtime{}}
\newcommand{\duedate}{Due: \labday, \assignmentduedate, \labyear{} at \labtime{}}

% Edit these commands to give the name to the main program

\newcommand{\mainprogram}{\lstinline{generator.py}}
\newcommand{\mainprogramsource}{\lstinline{generator/generator.py}}
\newcommand{\maintestsource}{\lstinline{generator/tests/test_generator.py}}

% Edit this commands to describe key deliverables

\newcommand{\reflection}{\lstinline{README.md}}

% Commands to describe key git tasks

\newcommand{\gitcommitfile}[1]{\command{git commit #1}}
\newcommand{\gitaddfile}[1]{\command{git add #1}}

\newcommand{\gitadd}{\command{git add}}
\newcommand{\gitcommit}{\command{git commit}}
\newcommand{\gitpush}{\command{git push}}
\newcommand{\gitpull}{\command{git pull}}

% Use this when displaying a new command

\newcommand{\command}[1]{``\lstinline{#1}''}
\newcommand{\program}[1]{\lstinline{#1}}
\newcommand{\url}[1]{\lstinline{#1}}
\newcommand{\channel}[1]{\lstinline{#1}}
\newcommand{\option}[1]{``{#1}''}
\newcommand{\step}[1]{``{#1}''}

\usepackage{pifont}
\newcommand{\checkmark}{\ding{51}}
\newcommand{\naughtmark}{\ding{55}}

\usepackage{listings}
\lstset{
  basicstyle=\small\ttfamily,
  columns=flexible,
  breaklines=true
}

\usepackage{fancyhdr}

\usepackage[margin=1in]{geometry}
\usepackage{fancyhdr}

\pagestyle{fancy}

\fancyhf{}
\rhead{Computer Science 280}
\lhead{Laboratory Assignment \assignmentnumber{}}
\rfoot{Page \thepage}
\lfoot{\duedate}

\usepackage{titlesec}
\titlespacing\section{0pt}{6pt plus 4pt minus 2pt}{4pt plus 2pt minus 2pt}

\newcommand{\labtitle}[1]
{
  \begin{center}
    \begin{center}
      \bf
      CMPSC 280\\Software Engineering\\
      Fall 2017\\
      \medskip
    \end{center}
    \bf
    #1
  \end{center}
}

\begin{document}

\thispagestyle{empty}

\labtitle{Laboratory \assignmentnumber{} \\ \assigneddate{} \\ \duedate{}}

\section*{Objectives}

The goal of this assignment is to investigate programming as a problem solving task, focusing on the necessity of
reflecting on the understanding on the problem, the plan chosen to solve the problem, and the execution of this plan.
You will be responsible for using each of the software tools that you created in the last two laboratory assignments,
identifying areas in which they can be improved. Next, you will continue to apply effective project management and team
work strategies to enhance and maintain these open-source projects that are available on GitHub. Using the ``GitHub
flow'' model, you will improve and evaluate these projects and the processes that governed their creation.

\section*{Suggestions for Success}

\begin{itemize}
  \setlength{\itemsep}{0pt}

\item {\bf Use the laboratory computers}. The computers in this laboratory feature specialized software for completing
  this course's laboratory  assignments. If it is necessary for you to work on a different machine, be sure to regularly
  transfer your work to a laboratory machine so that you can check its correctness. If you cannot use a laboratory
  computer and you need help with the configuration of your own laptop, then please carefully explain its setup to the
  systems administrator or the course instructor when you are asking questions.

\item {\bf Follow each step carefully}. Slowly read each sentence in the assignment sheet, making sure that you
  precisely follow each instruction. Take notes about each step that you attempt, recording your questions and ideas and
  the challenges that you faced. If you are stuck, then please tell the course instructor what assignment step you
  most recently completed.

\item {\bf Regularly ask and answer questions}. Please log into Slack at the start of a laboratory session and then join
  the appropriate channel. If you have a question about one of the steps in an assignment, then you can post it to the
  designated channel. Or, you can ask one of your team members or talk with course instructor in person or through
  Slack.

\item {\bf Store all your deliverables in GitHub}. As in past laboratory assignments, you will be responsible for
  storing all of your files (e.g., Python source code and Markdown-based writing) in a Git repository using GitHub and
  GitHub Classroom. Please verify that you have saved your source code in your Git repository by using \command{git
  status} to ensure that everything is updated. You can see if your assignment submission meets the established
  correctness requirements by using the provided checking tools on your local computer and in checking the commits in
  GitHub. Additionally, your team should perform all project management and communication tasks through GitHub. That is,
  all discussions about your programming systems product and the source code of and documentation for it must be
  available in GitHub.

\item {\bf Keep all of your files}. Don't delete your programs, output files, and written reports after you submit them
  through GitHub; you will need them again when you study for the quizzes and examinations and work on the other
  laboratory and final project assignments.

\item {\bf Back up your files regularly}. All of your files are regularly backed-up to the servers in the Department of
  Computer Science and, if you commit your files regularly, stored on GitHub. However, you may want to use a flash
  drive, Google Drive, or your favorite backup method to keep an extra copy of your files on reserve. In the event of
  any type of system failure, you are responsible for ensuring that you have access to a recent backup copy of all your
  files.

\item {\bf Hone your technical writing skills}. Computer science assignments require you to write technical
  documentation and descriptions of your experiences when completing each task. Take extra care to ensure that your
  writing is interesting and both grammatically and technically correct, remembering that computer scientists must
  effectively communicate and collaborate with both their team members and the course instructor.

\item {\bf Review the Honor Code on the syllabus}. While you may discuss your assignments with others, copying source
  code or writing is a violation of Allegheny College's Honor Code.

\end{itemize}

\vspace*{-1em}

\section*{Reading Assignment}

If you have not done so already, please read all of the relevant ``GitHub Guides'', available at
\url{https://guides.github.com/}, that explain how to use many of the features that GitHub provides. In particular,
please make sure that you have read guides such as ``Mastering Markdown'' and ``Documenting Your Projects on GitHub'';
each of them will help you to understand how to use both GitHub and GitHub Classroom. To do well on this assignment, you
should read Chapter 7 of SETP, paying particularly close attention to the types of internal and external documentation
that engineers should develop for their software. Finally, to review the steps associated with the ``GitHub flow''
model, you should review the web sites referenced in the last two assignments.

\section*{Reviewing the Prior Programming Systems Products}

In the past two laboratory assignments, students in this class implemented three programming systems products, that are
outlined in the following descriptions, which you should carefully review so as to ensure that you understand the
high-level goals for each of these software tools.

\subsubsection*{GatorGrouper: A Software Tool for Forming Teams of Programmers}

It is common for projects in computer science courses to involve teamwork. Since there is evidence that there are
downsides associated with allowing students to form their own teams, course instructors often implement bespoke software
tools to automatically creates these teams. However, these programs are not general-purpose, often requiring revision each
type an instructor teaches a new class. Moreover, these tools often resort to randomly forming groups, ignoring the
characteristics of the students in the course. These are the problems that GatorGrouper solves! Given access to
information about the students in a course, GatorGrouper automatically forms teams of programmers, using either random
algorithms or methods that strategically devise the groups.

\subsubsection*{Edurate: A Software Tool for Rating Software Educators}

While instructors at many colleges and universities are evaluated by students through student surveys at the end of a
semester, most institutions do not have a system for intermediate evaluations. Even if an instructor were to implement a
Google Form and store the results of surveys in Google Drive, these results are not yet fully analyzed by the supporting
software. This is the problem that Edurate solves! Given access to a spreadsheet of results from in-class survey ratings
completed by students, Edurate analyzes the numerical and textual data to give course instructors concrete and
actionable ideas for how to improve their teaching. Along with pointing out key numerical trends, Edurate uses natural
language processing methods, like those implemented in the Gensim Python library, to extract and present keywords and
topics about the instructor's teaching.

\vspace*{-.5em}

\subsubsection*{Accelergator: A Software Tool for Accelerated and Adaptive Advising}

Even though instructors at many colleges and universities are expected to provide academic advising to students, most
institutions do not provide any software to support this critical task. Even if an instructor were to implement a Google
Form to collect data and status updates from their advisees and store the results from these forms in Google Drive, they
would not be fully analyzed by the supporting software. This is the problem that Accelergator solves! Given access to a
spreadsheet of data points about the academic interests and standing of advisees, Accelergator analyzes the numerical
and textual responses to give advisors concrete and actionable ideas for how to adaptively advise their students. Along
with pointing out key numerical trends and checkpoint completion status, Accelergator uses natural language processing
methods, like those implemented in the Gensim library, to extract and present keywords and topics about an instructor's
advisees.

\vspace*{.25em}

\noindent
Students can access the current version of these three software tools by visiting these web sites:

\begin{enumerate}
  \item \url{https://github.com/GatorGrouper/gatorgrouper}
  \item \url{https://github.com/Edurate/edurate}
  \item \url{https://github.com/Accelegator/accelegator}
\end{enumerate}


\section*{Evaluating and Improving a Programming Systems Product}

Now that you understand these systems, you should identify at least three and no more than five members of the class who
will, going forward, serve as the maintainers for a project. Please note that no student can serve as the maintainer for
more than one of the projects. Each student who is a project maintainer will be responsible for overseeing and
maintaining their project, vetting the pull requests associated with any forthcoming defect fixes or enhancements.
Students who are not project managers are responsible for downloading and using all three of the software tools, raising
issues and creating pull requests for each of them. Students who are project managers only need to evaluate and try to
improve the two systems for which they are not a manager; for their own projects, these students only need to manage any
of the raised issues and pull requests.

Once you have talked to the customer and your classmates so as to best understand the purpose of each project, you
should visit their GitHub pages and follow their {\tt README.md} files to download, install, and use them, taking care
to start with a new repository. As you are completing this task, you should first determine whether or not there are any
major defects in the system's implementation or documentation. If there are major problems, then you should first raise
an issue in the project's issue tracker and then work to resolve this defect. Alternatively, if you notice that the
system requires certain minor enhancements, then you should also raise an issue and then work to create the new feature.
As you focus first on major concerns and then consider enhancements, please ensure that you follow the GitHub flow model
and coordinate with the project's managers.

You should evaluate the quality of the internal documentation for each project, focusing on the source code comments,
the use of meaningful variable names and statement labels, proper source code formatting, and the documentation of the
program's data sources and structures. Next, you should carefully study the project's external documentation, ensuring
that it is self-contained and that it clearly describes the problem, the methods used to solve the problem, and the
sources and structure of the data. Ultimately, you must ensure that all aspects of the program's documentation correctly
describe the system and serve to encourage and support the people who use it.

Next, you should study the test cases for each of the projects. Do these tests following a plan? Are there enough test
cases to adequately cover the source code? If there is a need for more test cases, then where do you suggest that they
be added? In addition to investigating these issues, you should determine whether or not there are redundant test cases,
some of which should now be removed from the GitHub repository. Finally, you should ensure that each project's
documentation explains how to run the provided test suite using the {\tt pytest} testing framework.

Please note that it is not the responsibility of the project managers to tell the students in the course how to improve
the three software projects. Instead, each student is responsible for downloading, installing, using, and fully
improving each project. The instructor will assess students according to both the number and quality of the issues that
they raised and the number and quality of the pull requests that they created and saw merged. For the project that they
are managing, the instructor will evaluate the project managers according to how well they handled the merging of pull
requests. Every student, including the project managers, should make a final assessment of each software project, using
their GitHub repository to answer the following questions:

\begin{enumerate}
  \item How would you rate the overall usefulness and quality of the software tool?
  \item When you started this assignment, what were the areas in greatest need of improvement?
  \item Now that you have finished this laboratory assignment, what is the next step for each project?
  \item Now that you have completed these tools, what have you learned about project management?
  \item If you could ``redo'' one aspect of completing these projects, what would it be? Why?
\end{enumerate}

\section*{Accessing the Laboratory Assignment on GitHub}

To access the laboratory assignment, you should go into the \channel{\#announcements} channel in our Slack team and find
the announcement that provides a link for it. Copy this link and paste it into your web browser. Now, you should accept
the laboratory assignment and see that GitHub Classroom created a new GitHub repository for you to access the
assignment's review materials and to store the completed version of your review. Specifically, to access your new GitHub
repository for this assignment, please click the green ``Accept'' button and then click the link that is prefaced with
the label ``Your assignment has been created here''. If you accepted the assignment and correctly followed these steps,
you should have created a GitHub repository with a name like
``Allegheny-Computer-Science-280-Fall-2017/computer-science-280-fall-2017-lab-6-gkapfham''. Unless you provide the
instructor with documentation of the extenuating circumstances that you are facing, not accepting the assignment means
that you automatically receive a failing grade for it.

You will use the \reflection{} file in this GitHub repository to report on all of the work that you completed and review
the work that your class members finished. Specifically, you should use the \reflection{} file to evaluate the
effectiveness of the class, commenting on who completed what work and how well the work was completed. Make sure that
you review all of the questions in the previous section of the assignment as you write your \reflection{}. Finally,
considering the definitions given in MMM, you should also assess the quality of the programming systems product that the
class submitted before the deadline. Please see the instructor with any questions about this task.

As in past assignments, you can click the ``Clone or download'' button and, after ensuring that you have selected
``Clone with SSH'', please copy this command to your clipboard. To enter your course directory you should now type
\command{cd cs280F2017}. By typing \command{git clone} in your terminal and then pasting in the string that you copied
from the GitHub site you will download all of the code for this assignment. Before starting the assignment, make sure
that the other students in the class can access to their own private repository and everyone understands what type of
content they should include in the \reflection{} file. You may talk with the course instructor if you have questions
about the purpose of the GitHub repository that only contains a single \reflection{} file.

\section*{Summary of the Required Deliverables}

Using a report that the instructor shares with you through the commit log in GitHub, you will privately received a grade
on this assignment and feedback on your deliverables. The instructor will also evaluate the effectiveness with which the
class completed the assignment. Ultimately, this laboratory assignment invites you to submit, using GitHub, the
following deliverables.

\vspace*{-.5em}

\begin{enumerate}

\setlength{\itemsep}{0in}

\item Stored in a \reflection{} file, a reflection on your contributions to the three projects and the contributions
  made by the other members of this class. Refer to the relevant sections of this sheet for details about what to write
  in your repository's \reflection{} file. Notably, you must thoughtfully evaluate your work and the work of the other
  class members, including source code and documentation, for all three of the projects that are the focus of this
  assignment.

\item A properly documented, well-formatted, and correct version of all of the source code needed to fulfill the
  requirements given in the issue tracker of the repository for all three projects.

\item Stored in the \reflection{} file of the project's GitHub repository, user documentation that handles all of the
  issues raised in the issue tracker of the GitHub repository for your projects.

\end{enumerate}

As previously mentioned, the instructor will also evaluate the effectiveness with which the class completed this
assignment. The instructor will assess team effectiveness as {\bf excellent} when:

\vspace*{-.5em}

\begin{enumerate}
  \setlength{\itemsep}{0pt}

  \item All team members regularly contribute, in a timely fashion, to the GitHub project through the issue tracker,
    pull requests, and the committing of source code and documentation.

  \item All team members contribute correct, useful Python source code and/or Markdown documentation that supports the
    creation of a working and valuable programming systems product.

  \item All team members participate in the classroom discussions, sharing status updates and ideas.

  \item All team members regularly communicate with others in a respectful and informative manner.

  \item All team members work with others and the instructor to resolve any differences that arise.

\end{enumerate}

\vspace*{-.5em}

The instructor will assess team effectiveness as {\bf good} when the majority of the team members fulfill the
aforementioned standards. The team's effectiveness will be assessed as {\bf average} when some, but not the majority, of
the team members make meaningful contributions to the project, according to these set standards. The course instructor
will assess team effectiveness as {\bf below average} when, as a whole, the team does not function according to the
above standards.

The instructor will assess product quality as {\bf excellent} when, by the established deadline, the products' GitHub
repository contains a full programming systems product that contains: clear and useful documentation, a comprehensive
and passing test suite that effectively covers the code base, and correct implementations of all, or the majority of,
the requirements listed in the issue tracker. The instructor will assess product quality as {\bf good} when all of the
aforementioned standards are met, but some of the requirements are not fully implemented or not implemented correctly.
The product's quality will be assessed as {\bf average} when some, but not the majority, of the requirements are
properly fulfilled. The course instructor will assess product quality as {\bf below average} when, as a whole, the
product does not feature new requirements or there is a regression in functionality.

The course instructor expects that each of the three projects will be fully completed and ready for use by instructors
and/or students in the Department of Computer Science. For all three of the projects, every student will receive the
same baseline grade for team effectiveness and product quality. However, an individual student grade will be adjusted
higher or lower according to their contribution's to the project, as reflected in the material that they and others
include in the \reflection{} file of the private GitHub repository and in the public repository of each project's GitHub
repository. Please see the instructor if you have questions about the grading of this assignment.

\end{document}
