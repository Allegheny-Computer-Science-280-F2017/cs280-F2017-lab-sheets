\documentclass[11pt]{article}

% NOTE: The "Edit" sections are changed for each assignment

% Edit these commands for each assignment

\newcommand{\assignmentduedate}{November 7}
\newcommand{\assignmentassignedate}{October 17}
\newcommand{\assignmentnumber}{Five}

\newcommand{\labyear}{2017}
\newcommand{\labday}{Tuesday}
\newcommand{\labtime}{2:30 pm}

\newcommand{\assigneddate}{Assigned: \labday, \assignmentassignedate, \labyear{} at \labtime{}}
\newcommand{\duedate}{Due: \labday, \assignmentduedate, \labyear{} at \labtime{}}

% Edit these commands to give the name to the main program

\newcommand{\mainprogram}{\lstinline{generator.py}}
\newcommand{\mainprogramsource}{\lstinline{generator/generator.py}}
\newcommand{\maintestsource}{\lstinline{generator/tests/test_generator.py}}

% Edit this commands to describe key deliverables

\newcommand{\reflection}{\lstinline{README.md}}

% Commands to describe key git tasks

\newcommand{\gitcommitfile}[1]{\command{git commit #1}}
\newcommand{\gitaddfile}[1]{\command{git add #1}}

\newcommand{\gitadd}{\command{git add}}
\newcommand{\gitcommit}{\command{git commit}}
\newcommand{\gitpush}{\command{git push}}
\newcommand{\gitpull}{\command{git pull}}

% Use this when displaying a new command

\newcommand{\command}[1]{``\lstinline{#1}''}
\newcommand{\program}[1]{\lstinline{#1}}
\newcommand{\url}[1]{\lstinline{#1}}
\newcommand{\channel}[1]{\lstinline{#1}}
\newcommand{\option}[1]{``{#1}''}
\newcommand{\step}[1]{``{#1}''}

\usepackage{pifont}
\newcommand{\checkmark}{\ding{51}}
\newcommand{\naughtmark}{\ding{55}}

\usepackage{listings}
\lstset{
  basicstyle=\small\ttfamily,
  columns=flexible,
  breaklines=true
}

\usepackage{fancyhdr}

\usepackage[margin=1in]{geometry}
\usepackage{fancyhdr}

\pagestyle{fancy}

\fancyhf{}
\rhead{Computer Science 280}
\lhead{Laboratory Assignment \assignmentnumber{}}
\rfoot{Page \thepage}
\lfoot{\duedate}

\usepackage{titlesec}
\titlespacing\section{0pt}{6pt plus 4pt minus 2pt}{4pt plus 2pt minus 2pt}

\newcommand{\labtitle}[1]
{
  \begin{center}
    \begin{center}
      \bf
      CMPSC 280\\Software Engineering\\
      Fall 2017\\
      \medskip
    \end{center}
    \bf
    #1
  \end{center}
}

\begin{document}

\thispagestyle{empty}

\labtitle{Laboratory \assignmentnumber{} \\ \assigneddate{} \\ \duedate{}}

\section*{Objectives}

The focus of this assignment is on applying effective project management and team work strategies to enhance an existing
open-source project that is available on GitHub. Using GitHub and the ``GitHub flow'' model, you will work in an
assigned team to specify, design, implement, test, document, release, and maintain a fully functioning programming
systems product that can be deployed into immediate use. During the first two-week phase of this laboratory assignment,
your team will collaboratively complete all of the features requested by the on-site customer. During the second
one-week phase you will maintain and enhance a team's programming system product.

\section*{Suggestions for Success}

\begin{itemize}
  \setlength{\itemsep}{0pt}

\item {\bf Use the laboratory computers}. The computers in this laboratory feature specialized software for completing
  this course's laboratory  assignments. If it is necessary for you to work on a different machine, be sure to regularly
  transfer your work to a laboratory machine so that you can check its correctness. If you cannot use a laboratory
  computer and you need help with the configuration of your own laptop, then please carefully explain its setup to the
  systems administrator or the course instructor when you are asking questions.

\item {\bf Follow each step carefully}. Slowly read each sentence in the assignment sheet, making sure that you
  precisely follow each instruction. Take notes about each step that you attempt, recording your questions and ideas and
  the challenges that you faced. If you are stuck, then please tell the course instructor what assignment step you
  most recently completed.

\item {\bf Regularly ask and answer questions}. Please log into Slack at the start of a laboratory session and then join
  the appropriate channel. If you have a question about one of the steps in an assignment, then you can post it to the
  designated channel. Or, you can ask one of your team members or talk with course instructor in person or through
  Slack.

\item {\bf Store all your deliverables in GitHub}. As in past laboratory assignments, you will be responsible for
  storing all of your files (e.g., Python source code and Markdown-based writing) in a Git repository using GitHub and
  GitHub Classroom. Please verify that you have saved your source code in your Git repository by using \command{git
  status} to ensure that everything is updated. You can see if your assignment submission meets the established
  correctness requirements by using the provided checking tools on your local computer and in checking the commits in
  GitHub. Additionally, your team should perform all project management and communication tasks through GitHub. That is,
  all discussions about your programming systems product and the source code of and documentation for it must be
  available in GitHub.

\item {\bf Keep all of your files}. Don't delete your programs, output files, and written reports after you submit them
  through GitHub; you will need them again when you study for the quizzes and examinations and work on the other
  laboratory and final project assignments.

\item {\bf Back up your files regularly}. All of your files are regularly backed-up to the servers in the Department of
  Computer Science and, if you commit your files regularly, stored on GitHub. However, you may want to use a flash
  drive, Google Drive, or your favorite backup method to keep an extra copy of your files on reserve. In the event of
  any type of system failure, you are responsible for ensuring that you have access to a recent backup copy of all your
  files.

\item {\bf Hone your technical writing skills}. Computer science assignments require to you write technical
  documentation and descriptions of your experiences when completing each task. Take extra care to ensure that your
  writing is interesting and both grammatically and technically correct, remembering that computer scientists must
  effectively communicate and collaborate with their team members and the tutors, teaching assistants, and course
  instructor.

\item {\bf Review the Honor Code on the syllabus}. While you may discuss your assignments with others, copying source
  code or writing is a violation of Allegheny College's Honor Code.

\end{itemize}

\vspace*{-1em}

\section*{Reading Assignment}

If you have not done so already, please read all of the relevant ``GitHub Guides'', available at
\url{https://guides.github.com/}, that explain how to use many of the features that GitHub provides. In particular,
please make sure that you have read guides such as ``Mastering Markdown'' and ``Documenting Your Projects on GitHub'';
each of them will help you to understand how to use both GitHub and GitHub Classroom. To do well on this assignment, you
should review Chapters 1 and 2 of the SETP textbook. You should also read Chapter 3 and 4 of SETP, paying particularly
close attention to the content about strategies for project management and requirements elicitation. To ensure that your
team understands what kind of product it must collaboratively implement, please review Chapter 1 in MMM, focusing on the
definition of a programming systems product. As your team begins to design your system during the first phase of this
assignment, please make sure that you review the Chapters 4 through 6 of SETP, carefully thinking about how you can use
technical diagrams to express the design of your system. Finally, to review the steps associated with the ``GitHub
flow'' model, you should read the web sites referenced in the last assignment.

% \vspace*{-.5em}

% \begin{itemize}
%   \setlength{\itemsep}{0pt}
%   % \item \url{https://guides.github.com/introduction/flow/}
%   \item \url{https://help.github.com/articles/github-flow/}
%   \item \url{https://help.github.com/categories/collaborating-with-issues-and-pull-requests/}
% \end{itemize}

% \vspace*{-1em}

\section*{Organizing Your Software Development Team}

You should organize yourselves into the teams given in the \channel{\#announcements} channel in our Slack team. Please
note that, in the first phase of this assignment you will be designing and implementing a programming systems product.
However, in the second phase, you will be enhancing and maintaining a software tool developed by another team. After
reviewing the reading assignments mentioned in the previous section and discussing the software that you are invited to
implement for this assignment, your team should discuss how it will complete the project tasks for the first phase of
the laboratory. As you are making task assignments, please think about the unique skillset of each member of your
development team. When it seems as though there are no team members who best fit certain roles, you should make
compromises to ensure that all work will still be successfully finished. If it is absolutely necessary to do so, your
team may also consult experts from the other team --- however, if you do this then you must carefully document all of
the high-level conversations that you have with these external experts and take care not to violate the Honor Code by
sharing source code or technical documentation. Your assignment of people to roles should ensure that individuals have
the opportunity to explore new tasks and learn more about many software engineering tasks. Please see the instructor if
you do not know how to complete these tasks.

Importantly, your team must complete this assignment in conjunction with the other students in your team. As such, it is
a good idea for you to pick project managers who will decide the best way to allocate tasks to any separate sub-teams
and team members. While it is up to your team to decide, it also may be a good idea for you to identify several project
managers who will agree to coordinate this project and approve all of the pull requests. Remember, it is important for
the project managers to ensure that all of their members know how to effectively use GitHub and then make a plan for how
you will control your source code and documentation and communication using GitHub features like the issue tracker,
commit log, and pull requests. Students who do not know how to use these features of GitHub should immediately schedule
a meeting with the course instructor and ask their project manager any questions about this matter. Remember, your grade
for this assignment is a function of the work that you completed and documented through the main project's GitHub
repository and reported on in your private repository's \reflection{}. Please see the course instructor if you do not
understand this grading strategy for this three-week assignment.

\section*{Accessing the Laboratory Assignment on GitHub}

To access the laboratory assignment, you should go into the \channel{\#announcements} channel in our Slack team and find
the announcement that provides a link for it. Copy this link and paste it into your web browser. Now, you should accept
the laboratory assignment and see that GitHub Classroom created a new GitHub repository for you to access the
assignment's starting materials and to store the completed version of your assignment. Specifically, to access your new
GitHub repository for this assignment, please click the green ``Accept'' button and then click the link that is prefaced
with the label ``Your assignment has been created here''. If you accepted the assignment and correctly followed these
steps, you should have created a GitHub repository with a name like
``Allegheny-Computer-Science-280-Fall-2017/computer-science-280-fall-2017-lab-5-gkapfham''. Unless you provide the
instructor with documentation of the extenuating circumstances that you are facing, not accepting the assignment means
that you automatically receive a failing grade for it.

You will use the \reflection{} file in this GitHub repository to report on all of the work that you completed and review
the work that your team and class members completed. Specifically, you should use the \reflection{} file to evaluate the
effectiveness of your team and the other team, commenting on who completed what work and how well the work was
completed. Considering the definitions given in MMM, you should also assess the quality of the programming systems
product that the class completed by the deadline. As in past assignments, you can click the ``Clone or download'' button
and, after ensuring that you have selected ``Clone with SSH'', please copy this command to your clipboard. To enter your
course directory you should now type \command{cd cs280F2017}. By typing \command{git clone} in your terminal and then
pasting in the string that you copied from the GitHub site you will download all of the code for this assignment. Before
starting the assignment, make sure that each of your team members has access to their own private repository and they
understand what type of content they should include in the \reflection{} file. Please see the instructor if you have
questions about the purpose of the GitHub repository that contains the \reflection{}.

Finally, note that each team will be assigned a separate project that they must complete in the first two weeks of this
assignment. Your team is responsible for creating a GitHub organization with your project name and then creating a
GitHub repository owned by the organization and also given the name of your project. Next, please make sure that every
member of your team is also a member of your GitHub organization and has appropriate access to the project's GitHub
repository, in accordance with your team's strategy for managing contributions. At the end of the first two weeks, it is
also critical for your team to add the members of the other team to your project's organization and repository. Then,
you will coordinate with this new team to enhance and maintain the system that you started. See the instructor
with questions about these tasks.

\section*{Collaborating With Your Software Development Team}

Your team should create a GitHub issue tracker entry that describes each of the key tasks that you need to complete for
this project. Attempting to manage risk and estimate the effort required for individual team members to resolve these
issues, you should now assign people to roles and tasks. While it is acceptable for you to have in-person discussions
with your team members, please remember that all important discussions and decisions must be documented in GitHub.
Finally, don't forget that you are completing this assignment in conjunction with the other teams in the course --- the
class should consider identifying project managers who will coordinate this project.

% Don't forget that your work will be evaluated according to the standards established
% on the final page of this assignment sheet. Please see the instructor if you do not know how to effectively collaborate.

Your team should work with the customer to prioritize the implementation of the issues now raised in the project's issue
tracker. When you are not sure how to implement a requirement, then you must clarify the matter with the customer and
with the other members of the team. Since each team member is member of the project's GitHub repository, everyone can
create ``branches'' allocated to, for instance, each of the program features that you want to implement or the
documentation that you want to write. Once a specific branch contains the finished version of its associated
deliverable, a team member should create a pull request for discussion. If a project manager judges that the pull
request has all of the expected characteristics, then it should be merged into the ``master'' branch of your repository.
If the pull request is not accepted, then the team member should improve it until it meets everyone's expectations. Your
team should continue to use this model, called ``GitHub flow'', to support the completion of all deliverables.

\section*{Summary of the Required Deliverables}

% This section gives an overview of the two projects that are the focus for this laboratory assignment.

During the first two weeks you will work with the on-site customer and your team members to specify, design, implement,
test, and document one of the following software systems. Then, in the second and final week of the assignment you will
obtain access to the other team's project and evaluate, enhance, and maintain it as necessary after consulting with the
on-site customer.

\vspace*{-.5em}

\subsubsection*{Edurate: A Software Tool for Rating Software Educators}

While instructors at many colleges and universities are evaluated by students through student surveys at the end of a
semester, most institutions do not have a system for intermediate evaluations. Even if an instructor were to implement a
Google Form and store the results of surveys in Google Drive, these results are not yet fully analyzed by the supporting
software. This is the problem that Edurate will solve! Given access to a spreadsheet of results from in-class survey
ratings completed by students, Edurate will analyze the numerical and textual data to give course instructors concrete
and actionable ideals for how to improve their teaching. Along with pointing our key numerical trends, Edurate will use
natural language processing methods, like those implemented in the Gensim Python library, to extract and present
keywords and topics about the instructor's teaching.

\vspace*{-.5em}

\subsubsection*{Accelergator: A Software Tool for Accelerated and Adaptive Advising}

Even though instructors at many colleges and universities are expected to provide academic advising to students, most
institutions to not provide any software to support this critical task. Even if an instructor were to implement a Google
Form to collect data and status updates from their advisees and store the results from these forms in Google Drive, they
would not be fully analyzed by the supporting software. This is the problem that Accelergator will solve! Given access
to a spreadsheet of data points about the academic interests and standing of advisees, Accelergator will analyze the
numerical and textual responses to give advisors concrete and actionable ideas for how to adaptively advise their
students. Along with pointing our key numerical trends and checkpoint completion status, Accelergator will use natural
language processing methods, like those implemented in the Gensim Python library, to extract and present keywords and
topics about an instructor's advisees.

Ultimately, this assignment invites you to submit, using GitHub, the following deliverables.

\vspace*{-.5em}

\begin{enumerate}

\setlength{\itemsep}{0in}

\item Stored in \reflection{}, a reflection on your contributions to the group project and the contributions made by
  your team members and the other members of this class. Refer to the relevant sections of this sheet for details about
  what to write in your repository's \reflection{} file. Notably, you must thoughtfully evaluate your work and the work
  of your team and non-team members for both of the projects that are the focus of this two-phased laboratory assignment.

\item A properly documented, well-formatted, and correct version of all of the source code needed to fulfill the
  requirements given in the issue tracker of the GitHub repository for your projects.

\item Stored in the \reflection{} file of the project's GitHub repository, user documentation that handles all of the
  issues raised in the issue tracker of the GitHub repository for your projects.

\end{enumerate}

\vspace*{-1em}

\section*{Evaluation of Your Laboratory Assignment}

Using a report that the instructor shares with you through the commit log in GitHub, you will privately received a grade
on this assignment and feedback on your deliverables. The instructor will also evaluate the effectiveness with which the
teams and the class completed the assignment.

The instructor will assess team effectiveness as {\bf excellent} when all of these standards are met:

\vspace*{-.5em}

\begin{enumerate}
  \setlength{\itemsep}{0pt}

  \item All team members regularly contribute, in a timely fashion, to the GitHub project through the issue tracker,
    pull requests, and the committing of source code and documentation.

  \item All team members contribute correct, useful Python source code and/or Markdown documentation that supports the
    creation of a working and valuable programming systems product.

  \item All team members participate in the classroom discussions, sharing status updates and ideas.

  \item All team members regularly communicate with others in a respectful and informative manner.

  \item All team members work with others and the instructor to resolve any differences that arise.

\end{enumerate}

\vspace*{-.5em}

The instructor will assess team effectiveness as {\bf good} when the majority of the team members fulfill the
aforementioned standards. The team's effectiveness will be assessed as {\bf average} when some, but not the majority, of
the team members make meaningful contributions to the project, according to these set standards. The course instructor
will assess team effectiveness as {\bf below average} when, as a whole, the team does not function according to the
above standards.

The instructor will assess product quality as {\bf excellent} when, by the established deadline, the product's GitHub
repository contains a full programming systems product that contains: clear and useful documentation, a comprehensive
and passing test suite that effectively covers the code base, and correct implementations of all, or the majority of,
the requirements listed in the issue tracker. The instructor will assess product quality as {\bf good} when all of the
aforementioned standards are met, but some of the requirements are not fully implemented or not implemented correctly.
The product's quality will be assessed as {\bf average} when some, but not the majority, of the requirements are
properly fulfilled. The course instructor will assess product quality as {\bf below average} when, as a whole, the
product does not feature new requirements or there is a regression in functionality.

Please note that your project's product quality will be assessed twice. After the completion of the first phase for this
assignment, the instructor and the team to receive the project will both assess the work, assigning a baseline grade for
the team that created the product. At the completion of the second and final phase of this assignment, the instructor
will assign another baseline grade for the product, bearing in mind its state before it was enhanced and maintained by
the second team. The first product quality grade will contribute to the grade of the first team to work on the project.
The second product quality grade will contribute to the grade for the second team.

Every student will receive the same baseline grade for team effectiveness and product quality. However, an individual
student grade will be adjusted higher or lower according to their contribution's to the project, as reflected in the
material that they and others include in the \reflection{} file of the private GitHub repository and in the public
repository of the project's GitHub repository.

\end{document}
